\documentclass{article}
\usepackage{hyperref}
\usepackage[utf8]{inputenc}

\title{Pandit Extended Abstract}
\author{Eric Gerard Moynihan}
\date{March 2022}

\begin{document}

\maketitle

\section{Introduction}
Pandit at its core is a \textbf{distributed service level proxy}.
This means it is a proxy tasked with interpreting data on a service level and caching it
in a distributed manner, allowing it to speed up read and write times of clients querying the services
in question.

Pandit is designed to be able to proxy any data source such as REST APIs or SQL databases.
It reads in user defined Protocol Buffers \cite{protobufs} that are extended to provide information on
how to parse, cache and return data to clients.

A single instance of the Pandit daemon will be ran on each host in a cluster of machines. When a client queries
a service, the daemon will first check the cache for the data. If it cannot find it, it will delegate the request
to an authoritative container in the cluster.

Since the mapping of



\begin{thebibliography}{9}
    \bibitem{protobufs}
    Profocol buffers documentation - \href{https://developers.google.com/protocol-buffers}{developers.google.com}.
\end{thebibliography}

\end{document}