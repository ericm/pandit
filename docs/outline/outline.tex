\documentclass[12pt]{article}
\usepackage[a4paper, total={6in, 8in}]{geometry}

\title{Pandit: A Service Level Distributed Proxy.}
\author{Eric Gerard Moynihan}
\date{November 2021}

\begin{document}

\maketitle

\section{Outline}
This project entails creating a proxy that runs on each host machine in a cluster, turning services in the cluster into widely available microservices. 

This will be done by providing a translation layer between the application layer of the containers making up a service and the clients and proxies.
This translation layer will provide protobufs responses in place of the traditional container responses.

Clients running on different hosts in the cluster will make the proxy aware that they are interested in certain services.
When a client sends a request to a service, the proxy will first look for cached data on the host. 
If it cannot find it, it will delegate the request to an authorative container in the cluster.

These Protobuf responses can then be distributed to relevant instances of the proxy running on other hosts for caching purposes.
Since the mapping between application layer and Protobuf is implimentation specific, there can be side-effects pre-programmed into the resulting Protobuf, such as marking some fields to be updated in the cache before being written to the authorative container(s).

This approach to a distributed proxy will allow for improved read/write performance within a cluster, as well as simplify distributed application development since every dependency can be interfaced via a universal API.


\end{document}
